\documentclass{paper}


\usepackage{epsfig}
\usepackage{graphicx}
\usepackage{amsmath}
\usepackage{amssymb}

\usepackage{color}
\usepackage{subcaption}
\usepackage{caption}




\usepackage{mathrsfs}
\usepackage{algpseudocode}
\usepackage{algorithm}

\usepackage{float}
\usepackage{mathtools}
\usepackage{mdwlist}
\usepackage{gensymb}
\usepackage{array}
\usepackage{multirow}
\usepackage[hmargin=3cm]{geometry}
\usepackage{boxedminipage}



\setlength{\parindent}{0pt}
\setlength{\parskip}{18pt}


\usepackage[latin1]{inputenc} 
\usepackage[T1]{fontenc} 

\usepackage{listings} 
\lstset{% 
   language=Matlab, 
   basicstyle=\small\ttfamily, 
} 






\renewcommand{\algorithmicforall}{\textbf{Foreach}}
\newcommand{\init}{\textbf{INIT }}
\newcommand{\pluseq}{\mathrel{+}=}
\newcommand{\asteq}{\mathrel{*}=}
\newcommand{\myto}{\textbf{TO }}
\newcommand*\colvec[3][]{
    \begin{pmatrix}\ifx\relax#1\relax\else#1\\\fi#2\\#3\end{pmatrix}
}
\newcommand{\myparagraph}[1]{\paragraph{#1}\mbox{}\\}
\DeclarePairedDelimiter\ceil{\lceil}{\rceil}
\DeclarePairedDelimiter\floor{\lfloor}{\rfloor}



\title{Computational Photography Assignment 3}



\author{Single Michael\\08-917-445}
% //////////////////////////////////////////////////


\begin{document}



\maketitle


\section*{Task 1}
    
\section*{Task 2}
\begin{itemize}
    \item Our given image $I$ is an $m$ by $n$
    \item Image $I$ is monochromatic, i.e. there is only one color-channel.
\end{itemize}




\begin{algorithm}[H]
\caption{Moving Average box filter}
\begin{table}[H]
  \begin{tabular}{@{}lll@{}}
    \textbf{Input:} & Grayscale \emph{Image} I resolution m by n  \\
    \textbf{Output:} & \emph{Image} $\hat{I}$   \\
  \end{tabular} 
\end{table}
\textbf{Procedures:} $getDimensions(Image)$  \\
\setlength{\fboxrule}{0pt} 
\begin{boxedminipage}{1.0\textwidth}
  \begin{algorithmic}[1]
      \State $ [h,w] = getDimensions(I)$
      \State $ r = \ceil{\frac{w-1}{2}}$
      \ForAll{$Pixel \thinspace p \in Image \thinspace I$}
        \State $ contribution = 0$
        \ForAll{$Pixel \thinspace p_n \in r-Neighborhood \thinspace \ \mathcal{N}_r(p)$}
            \State $ contribution = contribution + I(p+p_n)$
        \EndFor
        \State $ \hat{I}(p) = \frac{contribution}{m \cdot n}$
      \EndFor
  \end{algorithmic}
  \end{boxedminipage}
  \vskip1.5pt
\label{alg:boxfilter}
\end{algorithm}

\paragraph{Remarks:}

\begin{itemize}
    \item By pixels in the Algorithm $\ref{alg:boxfilter}$ we are referring to the coordinates of the pixel in the image. 
    \item $I(p)$ denotes accessing the pixel-(color)-values in the images at the position of the pixel $p$ in the image $I$.
    \item $\mathcal{N}_r(p)$ denotes the neighborhood with radius $r$ around a given pixel $p$. In the context of pixel-coordinates, think of it as a box-grid, centred at the pixel coordinates of $p$. This grid has a radius of r. This means there are $r$ neighbors (pixel-coordinates in the grid) below, on top, on the left and on the right of $p$.
    \item Our algorithm can easily be extended for color Images by simply applying the same algorithm to each color-channel separately.
    \item The assumption of being provided by a m by n can easily be extended for the case when $n \neq m$. This only will affect the computation of the radius $r$ in algorithm $\ref{alg:boxfilter}$. \\ Computing $\ceil{0.5 \cdot \left( \ceil{\frac{m-1}{2}} + \ceil{\frac{n-1}{2}}\right)}$ would be a valid option in order to compute $r$. 
    \item If $w$ (i.e. n) is odd, then $\ceil{\frac{w-1}{2}}$ is equal to $\frac{w-1}{2}$. 
\end{itemize}


\paragraph{Aysmptotic Complexity}


\section*{Task 3}
\section*{Task 4}
\section*{Task 5}




\section*{Task 6}



\end{document}